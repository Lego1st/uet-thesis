\chapter{Conclusions}

\section{Conclusions}

This project presents a new subject quality rating database considering local image quality assessment. Due to the lack of \enquote*{ground truth} quality of patches, we expect HMII to be a useful database for patch-based approaches. We also introduce a simple effective patch-based deep neural network that allows for feature learning and regression in an end-to-end framework. We believe that this proposed approach could achieve better result if we enlarge HMII database.

\section{Future work}

\textbf{HMII Database:} There are still some limitations on the proposed database to improve. 

\begin{itemize}
  \item Enlarge database to increase the number of image and them number of subject per image
  \item Generate more images to cover more type of distortions 
  \item Filter with different outlier detection methods
\end{itemize}  

\textbf{Image-Patch model:} In the future, we are planning to design more Image-Patch models and do experiments to evaluate on different databases. With DIPQA, there are some applications that we also consider: 
\begin{itemize}
  \item Applied in Image and Video Compression
  \item Associated a pooling state to compete with other Image and Video Quality Assessment (VQA) algorithms
  \item Improve current IQA/VQA algorithms
\end{itemize}  
