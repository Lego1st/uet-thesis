%% ----------------------------------------------------------------
%% Thesis.tex -- MAIN FILE (the one that you compile with LaTeX)
%% ---------------------------------------------------------------- 

% Set up the document
\documentclass[a4paper, 13pt, oneside]{Thesis}  % Use the "Thesis" style, based on the ECS Thesis style by Steve Gunn
\graphicspath{Figures/}  % Location of the graphics files (set up for graphics to be in PDF format)

% Include any extra LaTeX packages required
\usepackage[square, numbers, comma, sort&compress]{natbib}  % Use the "Natbib" style for the references in the Bibliography
%\usepackage[round]{natbib}  % Use the "Natbib" style for the references in the Bibliography
\usepackage{verbatim}  % Needed for the "comment" environment to make LaTeX comments
\usepackage{vector}  % Allows "\bvec{}" and "\buvec{}" for "blackboard" style bold vectors in maths
\usepackage{tabularx}
\usepackage{indentfirst}
\usepackage{float}
\usepackage{notoccite}
\usepackage[dvipsnames]{xcolor}
\hypersetup{allcolors=black, colorlinks=true}  % Colours hyperlinks in blue, but this can be distracting if there are many links.
\usepackage[linesnumbered,ruled]{algorithm2e}
\usepackage{pgfplots}
\usepackage{multirow}
\usepackage{microtype}
\usepackage{pdfpages}
\usepackage[]{algorithm2e}
\usepackage{csquotes}
\usepackage{subcaption}
\usepackage{tikz}
\usetikzlibrary{calc}
\newcommand{\norm}[1]{\left\lVert#1\right\rVert}

% Include any extra LaTeX packages required

%% ----------------------------------------------------------------
\begin{document}

\frontmatter      % Begin Roman style (i, ii, iii, iv...) page numbering

% Set up the Title Page
% \title  {MODELING HUMAN VISUAL SYSTEM IN PATCH-BASE IMAGE QUALITY ASSESSMENT USING DEEP LEARNING}
% \authors  {\texorpdfstring
            % {\href{nguyentrungnghia1812@gmail.com}{Nguyen Trung Nghia}}
            % {Nguyen Trung Nghia}
            %}
% \addresses  {\groupname\\\deptname\\\univname}  % Do not change this here, instead these must be set in the "Thesis.cls" file, please look through it instead
% \date       {\the\year}
% \subject    {}
% \keywords   {}

%\maketitle

\begin{titlepage}
	\begin{center}
		\bigskip
		%\setlength{\parskip}{2pt}
		{\mynorm\bf {\VNUNAME}\par}
		{\mynorm\bf {\UNIVNAME}\par}      
		\bigskip \bigskip \bigskip \bigskip
		\centerline{\includegraphics[width=3cm]{figures/coltech.png}}
		\bigskip \bigskip
		{\mybig \bf Nguyen Trung Nghia \par}
		\vfill
		{\mylarge \bf MODELING HUMAN VISUAL SYSTEM \par IN PATCH-BASE IMAGE QUALITY ASSESSMENT \par USING DEEP LEARNING \par}
		\vfill \vfill
		\bigskip \bigskip      
		{\mybig \bf Major: \maj \par}      
		\bigskip \bigskip \bigskip   \bigskip
		{\bf \LOCNAME \ - \the\year \par}   
	\end{center}
	\begin{tikzpicture}[remember picture, overlay]
	\draw[line width = 3pt] ($(current page.north west) + (2.0in,0.0in)$) rectangle ($(current page.south east) + (0.5in,2.0in)$);
	\draw[line width = 0.5pt] ($(current page.north west) + (2.05in,-0.05in)$) rectangle ($(current page.south east) + (0.45in,2.05in)$);
	\end{tikzpicture}
\end{titlepage}
\newpage
\begin{titlepage}
	\begin{center}
		{\mynorm\bf {\VNUNAME}\par}
		{\mynorm\bf {\UNIVNAME}\par}      
		\bigskip \bigskip \bigskip \bigskip \bigskip \bigskip
		{\mybig \bf Nguyen Trung Nghia \par}
		\bigskip \bigskip \bigskip \bigskip \bigskip \bigskip
		{\mylarge \bf MODELING HUMAN VISUAL SYSTEM \par IN PATCH-BASE IMAGE QUALITY ASSESSMENT \par USING DEEP LEARNING \par}
		\bigskip \bigskip \bigskip \bigskip \bigskip \bigskip
	\end{center}
	{\mybig \bf Major: \maj \par}      
	\bigskip
	\bigskip
	\bigskip        
	{\mybig \bf Supervisor: \supname \par}
	{\mybig \bf Co-Supervisor: \cosupname \par}    
	\vfill
	\begin{center}
		{\bf \LOCNAME \ - \the\year \par}   
	\end{center}
	\begin{tikzpicture}[remember picture, overlay]
	\draw[line width = 3pt] ($(current page.north west) + (2.0in,0.0in)$) rectangle ($(current page.south east) + (0.5in,2.0in)$);
	\draw[line width = 0.5pt] ($(current page.north west) + (2.05in,-0.05in)$) rectangle ($(current page.south east) + (0.45in,2.05in)$);
	\end{tikzpicture}
\end{titlepage}

%% ----------------------------------------------------------------

\setstretch{1.3}  % It is better to have smaller font and larger line spacing than the other way round

% Define the page headers using the FancyHdr package and set up for one-sided printing
\fancyhead{}  % Clears all page headers and footers
\rhead{}  % Sets the right side header to show the page number
\lhead{}  % Clears the left side page header
\fancyfoot[C]{\thepage}

\pagestyle{fancy}  % Finally, use the "fancy" page style to implement the FancyHdr headers

%% ----------------------------------------------------------------
% Declaration Page required for the Thesis, your institution may give you a different text to place here
\Declaration{

% \addtocontents{toc}{\vspace{1em}}  % Add a gap in the Contents, for aesthetics

\emph{
\say{I hereby declare that the work contained in this thesis is of my own and has not been previously submitted for a degree or diploma at this or any other higher education institution. To the best of my knowledge and belief, the thesis contains no materials previously published or written by another person except where due reference or acknowledgement is made.}
}
\bigskip
\bigskip
\bigskip

Signature .............................................................................

% -------------------------------- Begin comment
\iffalse
I, AUTHOR NAME, declare that this thesis titled, `THESIS TITLE' and the work presented in it are my own. I confirm that:

\begin{itemize} 
\item[\tiny{$\blacksquare$}] This work was done wholly or mainly while in candidature for a research degree at this University.
 
\item[\tiny{$\blacksquare$}] Where any part of this thesis has previously been submitted for a degree or any other qualification at this University or any other institution, this has been clearly stated.
 
\item[\tiny{$\blacksquare$}] Where I have consulted the published work of others, this is always clearly attributed.
 
\item[\tiny{$\blacksquare$}] Where I have quoted from the work of others, the source is always given. With the exception of such quotations, this thesis is entirely my own work.
 
\item[\tiny{$\blacksquare$}] I have acknowledged all main sources of help.
 
\item[\tiny{$\blacksquare$}] Where the thesis is based on work done by myself jointly with others, I have made clear exactly what was done by others and what I have contributed myself.
\\
\end{itemize}
 
 
Signed:\\
\rule[1em]{25em}{0.5pt}  % This prints a line for the signature
 
Date:\\
\rule[1em]{25em}{0.5pt}  % This prints a line to write the date
}
\fi
% ----------------------------- End comment

\clearpage  % Declaration ended, now start a new page

%% ----------------------------------------------------------------
% Approval
\Approval{

%\addtocontents{toc}{\vspace{1em}}  % Add a gap in the Contents, for aesthetics

\emph{
\say{I hereby approve that the thesis in its current form is ready for committee examination as a requirement for the Bachelor of Computer Science degree at the University of Engineering and Technology.}
}
\bigskip
\bigskip
\bigskip

Signature .............................................................................
}

\clearpage  % Declaration ended, now start a new page

%% ----------------------------------------------------------------
\iffalse
% The "Funny Quote Page"
\pagestyle{empty}  % No headers or footers for the following pages

\null\vfill
% Now comes the "Funny Quote", written in italics
\textit{``Write a funny quote here.''}

\begin{flushright}
If the quote is taken from someone, their name goes here
\end{flushright}

\vfill\vfill\vfill\vfill\vfill\vfill\null
\clearpage  % Funny Quote page ended, start a new page
\fi
%% ----------------------------------------------------------------

% Acknoledgement
\setstretch{1.3}  % Reset the line-spacing to 1.3 for body text (if it has changed)

% The Acknowledgements page, for thanking everyone
\acknowledgements{
	\addtocontents{toc}{\vspace{1em}}  % Add a gap in the Contents, for aesthetics
	
	
  First and foremost, I would like to express my sincere thanks to my supervisor Ph.D. Le Thanh Ha and M.Sc. Pham Thanh Tung for their support and guidance throughout this research work.
  
  I greatly appreciated the Department of Computational Science and Engineering, and HMI Lab both at the VNU UET for the generous support.
  
  I would also like to thanks my teachers and friends for their support throughout my time in
  University of Engineering and Technology, Vietnam National University, Hanoi
  
  This thesis was partly supported by the University of Fire Fighting and Prevention under the subjective image quality experiment and InfoRe Technology for providing working environment. 
	
	
}

\clearpage  % End of the Acknowledgements
%% ----------------------------------------------------------------

\myabstract{
  \addtocontents{toc}{\vspace{1em}}
  
  As humans are the ultimate receivers of the majority of visual signals being processed, the most accurate way of assessing image quality is to ask humans for their opinions of an image’s quality, known as the subjective image quality assessment (IQA). The subjective image quality scores gathered from all subjects are processed to be the mean opinion score (MOS), which is regarded as the ground truth of image quality. Conventionally, a number of full-reference image quality assessment (FR-IQA) methods adopted various computational models of the human visual system (HVS) from psychological vision science research.
  
  The image compression is one of the most
  prominent applications that require IQA metrics to be highly
  correlated with human vision. To explore IQA algorithms that are
  more consistent with human vision, several calibrated databases
  have been constructed. However, the distorted images in the existing databases are usually generated by corrupting the pristine
  images with various distortions in coarse levels, such that the
  IQA algorithms validated on them may be inefficient to optimize
  the image compression with fine-grained quality
  differences. In addition, HVS is differently sensitive to features of image patch, the \enquote*{ground truth} quality of patch is essential for training patch-based methods, but in practice it’s easy to obtain the ground truth quality of images
  rather than patches. 
  
  So an experimental quality assessment to approach database for image patch has been developed. We propose Full-reference Deep Image-Patch Quality Assessment (DIPQA), a novel image-patch quality assessment that used deep neural network to estimate the \enquote*{ground truth} for patches with the developed database.
  
  Seven well-know IQA algorithms are evaluated and analyzed on the proposed database to show that there is still large room for improvement regarding fine-grained patch-based method. In the following experiment, we train and evaluate the proposed DIPQA on the proposed database and show competitive performance to other FR methods. DIPQA is expected to improve the performance of many applications that require patch's \enquote{ground truth} especially in image compression. 
}
% The Abstract Page
% \addtotoc{Abstract}  % Add the "Abstract" page entry to the Contents
%\abstract{
%\addtocontents{toc}{\vspace{1em}}  % Add a gap in the Contents, for aesthetics
%\textbf{Abstract:}
%
%  
%}
%\thispagestyle{empty}
%\clearpage  % Abstract ended, start a new page
%% ----------------------------------------------------------------

%\textsc{\Vabstract {
%  Con người là nguồn thu nhận tín hiệu hữu hình khá chính, do đó cách chính xác nhất để đánh giá chất lượng của hình ảnh là nhờ con người chấm điểm, điểm này được gọi là điểm đánh giá chủ quan chất lượng (IQA). Điểm đánh giá chủ quan được thu thập và tính trung bình của nhiều người (MOS) và được gọi là điểm chất lượng ảnh thực sự. Để đáp ứng nhu cầu mô phỏng được hệ thống trực quan của con người (HVS) trong các nghiên cứu về tâm lý thị giác, một loạt các phương pháp đánh giá chất lượng ảnh dựa vào ảnh tham chiếu (FR-IQA) đã được phát triển.
%
%  Nén ảnh là một trong những ứng dụng phổ biến và yêu cầu điểm IQA phải có tương quan lớn với thị giác người. Một số cơ sở dữ liệu đã được xây dựng nhằm phục vụ nghiên cứu những thuật toán IQA phù hợp với thị giác người. Tuy nhiên, những ảnh nhiễu trong các bộ dữ liệu hiện tại có đa dạng loại nhiều nhưng chỉ áp dụng nhiều ở dạng hạt thô (coarse level), điều này khiến cho các thuật toán IQA không thực sự hiệu quả trong việc tối ưu nén với các dạng nhiễu hạt mịn (fine-grained). Thêm nữa, HVS khá nhạy cảm với các mảng vá nhỏ (image patch), điểm thực sự cho các mảnh này cũng rất cần thiết cho các phương pháp tiếp cận vá (patch-based), tuy vậy thì hiện chưa có dữ liệu cho các mảnh vá này.
%  
%  Một thí nghiệm đánh giá chất lượng ảnh cho các mảnh vá nhằm xây dựng một bộ dữ liệu đã được thực hiện. Sau đó chúng tôi đề xuất một mô hình học sâu sử dụng ảnh tham chiếu Deep Image-Patch Quality Assessmen (DIPQA), mô hình sử dụng mạng nơ ron học sâu để dự đoán điểm thực sự cho các mảnh vá với dữ liệu đã được tạo ra trong dự án này.
%  
%  Bảy thuật toán IQA đã được đánh giá và phân tích trên tập dữ liệu đề xuất, cho thấy rằng vẫn có nhiều thứ để cải thiện cho các phương pháp đánh giá mảnh vá hạt nhiễu. Sau đó, chúng tôi huấn luyện và đánh giá mô hình đề xuất DIPQA trên tập dữ liệu, mô hình đem lại kết quả tốt hơn hẳn so với các phương pháp tham chiếu được sử dụng. DIPQA được hi vọng sẽ có thể nâng hiệu năng của các ứng dụng có sử dụng điểm mảnh vá, đặc biệt là trong nén ảnh.
%	
%}
%\thispagestyle{empty}
%\clearpage}


%% -----------------------------------------------------------

\pagestyle{fancy}  %The page style headers have been "empty" all this time, now use the "fancy" headers as defined before to bring them back


%% ----------------------------------------------------------------
\lhead{\emph{Contents}}  % Set the left side page header to "Contents"
\tableofcontents  % Write out the Table of Contents

%% ----------------------------------------------------------------
\lhead{\emph{List of Figures}}  % Set the left side page header to "List if Figures"
\listoffigures  % Write out the List of Figures

%% ----------------------------------------------------------------
\lhead{\emph{List of Tables}}  % Set the left side page header to "List of Tables"
\listoftables  % Write out the List of Tables

%% ----------------------------------------------------------------
\setstretch{1.5}  % Set the line spacing to 1.5, this makes the following tables easier to read
\clearpage  % Start a new page



\lhead{\emph{Abbreviations}}  % Set the left side page header to "Abbreviations"
\listofsymbols{ll}  % Include a list of Abbreviations (a table of two columns)
{
% \textbf{Acronym} & \textbf{W}hat (it) \textbf{S}tands \textbf{F}or \\
%\textbf{LAH} & \textbf{L}ist \textbf{A}bbreviations \textbf{H}ere \\
IQA & Image Quality Assessment \\
CNN & Convolutional Neural Network \\
DIPQA & Deep Image-Patch Quality Assessment \\
DMOS & Differential Mean Opinion Score \\
FR & Full-reference \\
MOS & Mean Opinion Score \\
NR & No-reference \\
RR & Reduced-reference \\
}

%% ----------------------------------------------------------------
\clearpage  % Start a new page

%% -------------- COMMENT OUT
\iffalse 
\lhead{\emph{Physical Constants}}  % Set the left side page header to "Physical Constants"
\listofconstants{lrcl}  % Include a list of Physical Constants (a four column table)
{
% Constant Name & Symbol & = & Constant Value (with units) \\
Speed of Light & $c$ & $=$ & $2.997\ 924\ 58\times10^{8}\ \mbox{ms}^{-\mbox{s}}$ (exact)\\

}

%% ----------------------------------------------------------------
\clearpage  %Start a new page
\lhead{\emph{Symbols}}  % Set the left side page header to "Symbols"
\listofnomenclature{lll}  % Include a list of Symbols (a three column table)
{
% symbol & name & unit \\
$a$ & distance & m \\
$P$ & power & W (Js$^{-1}$) \\
& & \\ % Gap to separate the Roman symbols from the Greek
$\omega$ & angular frequency & rads$^{-1}$ \\
}
%% ----------------------------------------------------------------
% End of the pre-able, contents and lists of things
% Begin the Dedication page

\setstretch{1.3}  % Return the line spacing back to 1.3

\pagestyle{empty}  % Page style needs to be empty for this page
\dedicatory{For/Dedicated to/To my\ldots}


\fi
%% ------------END COMMENT--------------------------------------

\addtocontents{toc}{\vspace{2em}}  % Add a gap in the Contents, for aesthetics


%% ----------------------------------------------------------------
\mainmatter	  % Begin normal, numeric (1,2,3...) page numbering
\pagestyle{fancy}  % Return the page headers back to the "fancy" style
\fancyhf{}
\fancyhead[RE,LO]{\leftmark}
\fancyfoot[C]{\thepage}
\setlength{\parindent}{2em}
% Include the chapters of the thesis, as separate files
% Just uncomment the lines as you write the chapters

% \input{Chapters/ChapterTest} % Introduction


\input{Chapters/Introduction} % Introduction

\chapter{Background}

\section{Image Quality Assessment}

As portrayed in the introduction, image quality assessment (IQA) is vital for many applications, but on the other hand is hard to achieve as such because of its reliance on the quantification of human perception. 
The most solid approach to embrace IQA is through subjective assessments, but this is not practical in real-life applications since users can’t always be dependent upon to comment on the perceived quality. 
On the other hand, objective image quality assessment focuses on implementing human perception models that can estimate the quality of an image as perceived by a person based solely on pixel analysis information.

In the following, we briefly review current subjective quality assessment methods to then go deeper in the state-of-the-art of methods for objective quality assessment.

\subsection{Subjective image quality assessment}

Subjective image quality assessment methods use human observers to express their personal opinion on the quality of images which are used to be assessed. 
Because humans are the end users in a large portion of the multimedia applications, subjective IQA methods are the most reliable and accurate for image quality assessment.

Several international standards have been proposed for performing subjective image quality assessment such as , ITU P913 \cite{ITU2014}, ITU P910 \cite{ITU-TRecommendationP.9102008} and ITU BT 500 \cite{Bt2002}. 
The main objective of subjective IQA methods for a given set of images is to assign a score to each of them that quantifies the perceived quality of the user. In most cases, a scaling process can be achieved, either explicitly or implicitly.

Subjective testing usually focuses on quantifying average observer's perceived quality. 
A group of subjects is requested to evaluate an image and give its perceived quality score.
These scores are then accumulated and the final score is calculated to reflect the quality perceived by an average observer. 
For the calculation of this final score, different scales could be used, for example, direct scaling in which the perceived quality of an image is calculated as the mean of the scores assigned to that image by each subject. (Mean Opinion Score (MOS) or Differential Mean Opinion Score (DMOS)). 
The objective IQA methods (to be followed) are intended to use different models to predict these mean values.

Despite being the most accurate and reliable, subjective IQA methods are highly impractical for real-world applications as it is very expensive and time-consuming to gather an adequate number of observers to evaluate image quality. Consequently, more practical objective IQA methods are used for many applications.

\subsection{Objective image quality assessment}

Rather than using human observers, objective IQA methods are aimed at using relevant models that can predict image visual quality as perceived by humans. Because these algorithms require no human observer, they are fast and very practical for many applications in the real world, such as image enhancement, image restoration, etc.

To estimate the perceptual quality of the given image (called test image), either in the presence or absence of its reference image, most of the objective IQA methods share a common framework of three main phases as illustrated in Fig.\ref{fig:obj-fw}. These
three phases are described in the following

\begin{figure}[H]
  \includegraphics[width=\linewidth]{figures/objective-framework.png}
  \caption{General Objective image quality assessment framework}
  \label{fig:obj-fw}
\end{figure}

\begin{enumerate}
  \item The test image is processed pixel by pixel or region by region in accordance with the objective IQA method used to measure the amount of distortion present in it. This phase then outputs the distortion measured in the form of a distortion map containing the image quality local description. This step is equivalent to the feature extraction. 
  \item A multidimensional phase is produced in the first phase, but humans perceive image quality as a single global entity rather than the local properties of an image. A spatial pooling strategy is generally used to downsample the multidimensional distortion map to a single quality score in order to produce a global quality assessment. \cite{Wang2006}
  \item Since in the first two phases non-linearity, which characterizes perception, is not used, the output may not be sufficiently accurate. Thus, to increase the overall accuracy of the framework, an appropriate strategy could be applied. This requires a set of images along with their subjective quality scores (obtained through subjective testing), and a parametric model whose parameters are learned through the analysis of image model predictions and their actual subjective scores through regression. This learned model is then used to transform predicted scores into better estimates allegedly consistent with human perception. 
\end{enumerate}

Objective IQA strategies are classified into three large categories.

\subsubsection{Full-reference image quality assessment (FR-IQA)}

FR-IQA methods aim to achieve objective IQA goals while taking as input both reference and test images. Because these algorithms also require reference images to estimate visual quality, their scope is limited to a few applications where reference images are easily available, such as compression of images and watermarking. 

Over time, a lot of FR-IQA algorithms were proposed. According to one of these methods, image quality can be computed as a peak-signal-to-noise ratio (PSNR), which is simply a ratio of a signal's maximum power and distortion power. The distortion power is generally calculated to calculate the pixel-wise difference between the reference and the distorted image in terms of mean-square-error (MSE). PSNR has the advantages of being simple and very inexpensive computationally, but it does not deliver very good performance because the essential physiological and psychophysical characteristics of the human visual system (HVS) are not included in this algorithm. 

Another FR-IQA algorithm, the Structural Similarity Index (SSIM) \cite{Wang2004}, advances FR-IQA from raw pixels to structures. It is based on the assumption that HVS is highly adapted to extract structural information present in an image, and degradation of images is perceived as a change in this structural information. SSIM therefore aims to evaluate the quality of an image by measuring variations in the structural information of distorted images (in relation to their reference image). In evaluating the perceptual quality of images, SSIM has been shown to outperform PSNR 

Another lately proposed FR-IQA \cite{Zhang2011} algorithm is the Function Similarity Index (FSIM). It is based on the fact that HVS uses low-level features to understand images (like edges and zero crossing). FSIM uses two features to estimate an image's quality: a primary feature called Phase Congruency, which is a contrast-invariant dimensionless measure of the local structure's significance, and an image gradient magnitude feature. FSIM shows superior performance on different datasets than PSNR and SSIM algorithms. 

Recently, Bosse \emph{et al.} \cite{Bosse2018} presents an IQA data-driven approach based on deep neural networks. The network consists of 10 convolution layers and 5 pooling layers for extraction of features, and 2 fully connected layers for regression, making it significantly deeper than related IQA models. Unique features of the proposed architecture are that I it can be used in a no-reference (NR) as well as in a FR-IQA setting with slight adaptations and (ii) it enables joint learning of local quality and local weights in a unified framework, i.e. the relative importance of local quality to the global quality estimate. 

\subsubsection{Reduced-reference image quality assessment (RR-IQA)}

RR-IQA methods aim to achieve objective IQA goals by estimating the quality of the test image while using partial reference image information. Usually this partial information is in the form of features extracted from the images of the reference. 

In communication networks, RR-IQA finds its application that is used to transmit images and videos. Using RR-IQA algorithms, partial reference image information transmitted through these communication networks can be used to track visual quality degradation of images and videos transmitted. In similar applications, therefore, RR-IQA algorithms are preferred over FR-IQA algorithms as presented in \cite{Atsawaraungsuk2015, Redi2010}.

\subsubsection{No-reference image quality assessment (NR-IQA)}

NR-IQA methods are intended to achieve the objectives of objective IQA by using only test images to estimate the quality of the image. Due to the lack of information on reference images, these methods are considerably more challenging than FR-IQA and RR-IQA. But due to their application in the wide variety of fields, they are also more desirable, ranging from image processing to image enhancement, where reference images are usually not available. NR-IQA methods are also used in a wide range of online applications, such as communication systems, image acquisition systems, etc. \cite{Chandler2013}, making it very important for them to be computationally cheap. 

Some early NR-IQA attempts used distortion-specific methods that approach IQA tasks by using models very specific to a specific type of distortion. These methods are more specific to applications where there is prior knowledge of the type of distortion. For example, in an application to measure quality losses in compressed images, knowledge of the appearance of compression artifacts, such as blocking and ringing, could be used to design NR-IQA methods that can detect their visibility 

It is more useful to have algorithms, regardless of the types of distortion, that can be applied for general purpose NR-IQA. Existing NR-IQA approaches for general purposes could be further divided into two broad categories: Natural scene statistic based approaches (NSS) and Feature learning based approaches


\subsection{IQA in Visual Data Compression}

\begin{figure}[H]
  \includegraphics[width=\linewidth]{figures/jpeg.png}
  \captionsetup{justification=raggedright}
  \caption{Examples of quantization table and the corresponding compressed
JPEG images.} 
  \label{fig:jpeg}
  \subcaption*{(a) JPEG default quantization table at quality factor equal to
50; 
    \newline(b) Optimized quantization table with the optimization goal of MS-SSIM; \newline(c) JPEG image using default quantization table at QF = 10, 0.234 bbp,
PSNR = 30.45, SSIM = 0.819, MS-SSIM = 0.946; \newline(d) JPEG image using
MS-SSIM optimized quantization table, 0.226 bpp, PSNR = 30.49, SSIM = 0.818, MS-SSIM = 0.953}

\end{figure}

While the bitstream has been normalized by image coding standards, different coding parameters or modes determined by different IQA metrics will obviously result in distinct compression performance. In JPEG, the custom quantization table is one of the optional coding parameters, and the default table is empirically determined based on human perception \cite{Wallace1992}. For example, in Fig.\ref{fig:jpeg}(a), which is scaled to generate quantization tables for other quality factors, the quality factor (QF) quantization table of the luminance component equal to 50 is shown. In addition to the standard JPEG quantization table, the open source and well-optimized JPEG codec, \emph{libjpeg}, has adopted another 8 quantization tables, one of which is an optimized MS-SSIM-based quantization table as shown in Fig.\ref{fig:jpeg}(b). 

In \cite{Ratnakar2000}, the researchers proposed optimization of the image-dependent quantization table based on the signal fidelity-based metric, MSE, which achieved substantial bit rate savings at the same quality as PSNR. However, because of the poor correlation between perceptual quality and PSNR, these optimization strategies can not ensure the same visual quality improvement. The researchers introduced SSIM and its variants into image and video coding in \cite{Channappayya2008}, \cite{Wang2012} and \cite{Ou2011} to optimize the process of rate distortion, but the performance improvement is not yet so satisfying. The upper and lower boundaries of the average SSIM index were derived for the first time by Channappayya \emph{et al.}\cite{Channappayya2008} as a function of the quantization rate for different source distributions, e.g. uniform, Gaussian and Laplacian. Wang \emph{et al.}\cite{Wang2012} used SSIM as the quality metric for optimizing rate distortion instead of MSE and achieved a bit rate saving of about 5\%-10\% compared to the original H.264/AVC. Ou \emph{et al.} \cite{Ou2011} applied SSIM to the problem of perceptual rate control with a gain of 0.008 SSIM (corresponding to a saving of 14\ percent bitrate). We can see from this work that the improvements in quality are still small. 

Essentially, with regard to the compression of the perceptual image, although different encoding optimization strategies can improve the image quality at the same bit rate level, the quality fluctuations are usually limited within a small range. However, most traditional IQA databases contain only coarse-grained compression distortion levels, and IQA algorithms on the fine-grained quality prediction for image compression problem can not be evaluated well. For example, the scaled quantization tables in Fig.\ref{fig:jpeg}(a) and \ref{fig:jpeg}(b), respectively, compress the JPEG images in Fig.\ref{fig:jpeg}(c) and \ref{fig:jpeg}(d) at similar bitrates. Although the image shows less blocking artifacts in Fig.\ref{fig:jpeg}(d), it has a lower SSIM value but higher PSNR and MS-SSIM values than the image shown in Fig.\ref{fig:jpeg}(c). These different IQA algorithms show opposite quality rankings on the level of fine-grained distortion, which motivates us to re-examine existing IQA algorithms and examine their suitability to distinguish fine-grained distortions 

\section{Neural Networks}

\subsection{Artificial Neural Network}

Artificial Neural Network (ANN) is not a brand new idea. It was first introduced as a computational model of "nerve net" in the human brain by Warren McCulloch and Walter Pitts \cite{McCulloch1943} in 1943. After that, the concept and architecture of neural networks are further developed by follow-up researchers. Neural networks have long been constrained by hardware performance. The advancement in GPU design and brain science led to a boom in the development of neural networks only in recent decades. 

\begin{figure}[H]
  \includegraphics[width=\linewidth]{figures/ann.jpg}
  \caption{The basic structure of Neural Network.}
  \label{fig:ann}
\end{figure}

A common modern neural networks consist of a large number of nodes called neurons. Each neuron does a simple calculation, usually $y = Wx + b$, where $W$ is called Weight and $b$ is called Bias. The neurons form multiple layers and the result value $y$ of each neuron is then passed to the neurons in the next layer. The first layer is called input layer as shown in Fig.\ref{fig:ann}. As its name implies, it takes features from outside the network as input. The last layer is output layer and its output value is the prediction given by the neural network.

\subsection{Training Neural Network}

First, Weight Metric and Bias Metric are initialized with random values (or pre-trained value obtained from other benchmark data). A training of the neural networks is required to adjust those parameters to fit into a particular task.

The most common and popular method of training neural network is back-propagation (BP) \cite{Rumelhart1986}. The goal of back-propagation is to compute the partial derivative, or gradient, $\frac{\partial E}{\partial w}$ of a loss function E with respect to any weight w in the network. The loss function
$E$ calculates the difference between prediction of neural network and its expected output, after one or a batch of sample data go through the network. A loss function is usually defined as:

\begin{equation} \label{eq:lossabs}
E=\frac{1}{N}\sum_{i=1}^{N}|f(x_i)-y_i| 
\end{equation}

or

\begin{equation} \label{eq:losssqr}
E=\frac{1}{N}\sum_{i=1}^{N}(f(x_i)-y_i)^{2} 
\end{equation}

where $f(x)$ is the equivalent function fo the neural network. Equation \ref{eq:lossabs} is called $L1$ loss while equation \ref{eq:losssqr} is called $L2$ loss. In practice, $L2$ loss is the most popular one because it is more sensitive to examples that far away from expected output. Thus
the trained neural network is hopefully more general. On the other hand, $L1$ loss is not
that sensitive to a minority of output that far from the expectation and takes care of the
average error of the majority. It is especially useful when training data is not very carefully
collected and may contain incorrect samples.

Thus the progress of training by BP can be presented as:

\begin{enumerate}
  \item Put one batch of training data through the neural network
  \item Calculate the loss between output and ground truth
  \item Go backward the network and calculate the partial derivative, or gradient, $\frac{\partial E}{\partial w}$
of loss function $E$ with respect to each weight $w$ in the network
  \item Update the weights in the network according to loss, gradient and learning rate (LR)
  \item Repeat step 1 to 4 until training ends, usually when a certain number of cycles set
by researcher is reached or the loss value is smaller than a threshold
\end{enumerate}

This method is called \enquote{back-propagation} partly because the partial derivative is calculated using chain rule:

\begin{equation}
\frac{\partial E}{\partial w_{ij}} = \frac{\partial E}{\partial o_j} \frac{\partial o_j}{\partial net_j} \frac{\partial net_j}{\partial w_{ij}}
\end{equation}

where $E$ is the loss function, $w_{ij}$ is the weight from neuron $i$ to neuron $j$, $o_j$ is the output of the neuron $j$, $net_j$ is the weighted sum of outputs to neuron $j$ from the previous layer.

We can calculate them one by one from the output layer to input layer and use the result in later layers for calculating the former layers. Therefore, calculating partial derivative is actually quite cheap when doing backward.

A neural network with multiple layer structure has proved its power on image recognition \cite{Rowley1998}. However, it suffers from \enquote{the curse of dimensionality} heavily. It means that the number of parameters in the network goes up quickly when the dimension (resolution) of input image increases. Early neural networks work on low resolution images such as $20 \time 20$ and $32 \time 32$. Early benchmark datasets, MNIST \cite{LeCun1998} and CIFAR10 \cite{Krizhevsky2009} for instance, are also collections of small images with $20 \time 20$ and $32 \time 32$ pixels. At that time, neural networks can take care of those images with hundreds or thousands of parameters. When the size of target image rise to around $200 \time 200$, an input layer with 40000 neurons is needed. Assuming the first hidden layer is fully connected and has the same number of neurons as the input layer, which is quite common in practice, at least $40000 \time 40000$ individual parameters is needed in just 2 layers. The mass of parameters not only consumes computing resources, but also causes serious overfitting problems. 

Overfitting means that a statistical model tries to describe each training sample rather than to find out regular patterns among the sample collection. Fig.\ref{fig:overfit} shows a simple case of overfitting. The regression function tends to "remember" the distinguishing features of each sample individually but fails to figure out the trend of all samples. Although it passes every sample point and has 0 loss, it is not generalizable to unseen data. Even a linear function has more prediction power than it. 

\begin{figure}[H]
  \centering
  \includegraphics[width=.7\linewidth]{figures/overfit.png}
  \caption{The function overfit the train samples.}
  \label{fig:overfit}
\end{figure}

Overfitting usually happens when neural network has too many parameters compared to the number of training samples, which enables the network easily remembering all samples. To limit the number of parameters in neural network, researchers find a way to reuse parameters in different parts of the image, which is a Convolutional Neural Network (CNN)

\subsection{Convolutional Neural Network}

A CNN is one type of neural network that specially designed for image recognizing. The architecture of CNN comes from the organization of animal visual cortex. Fig.\ref{fig:cnn} shows the basic structure of a convolutional neural network. $X_{i,j}$ represents the pixels in the input image. A is the kernel of the first convolutional layer and is repeatedly used on each block of four pixels. The neurons on the second layer then takes the outputs of the first layer as their inputs and use the same kernel B. Fig.\ref{fig:kernel} shows the mapping between 2 layers. One blocked in the front layer, which is a  $m \times n \times d_1$ tensor ($m = n$ in most cases), is multiplied by a $m \times n \times d_1 \times d_2$ kernel and mapped to a $1 \times 1 \times d_2$ block in the next layer. The $m \times n \times d_1$ block in the front layer is called receptive field, which means all neurons in such block is connected to one neuron in the next layer, and their information is gathered together by a neuron in next layer.

\begin{figure}[H]
  \centering
  \includegraphics[width=\linewidth]{figures/cnn.png}
  \caption{CNN share the kernel on each layer.}
  \label{fig:cnn}
  \includegraphics[width=0.5\linewidth]{figures/kernel.png}
  \caption{Kernel that maps $m \times n \times d_1$ block in the previous layer to an $1 \times 1 \times d_2$ block in next layer.}
  \label{fig:kernel}
\end{figure}

CNN is proved to be very efficient in pattern recognition and other image classification tasks. Its superior performance comes from some particular features. The most important feature is perhaps its spatial invariant. Since the same kernel is used repeatedly in the whole input space, it can detect its corresponding pattern no matter where the pattern shows up. This feature significantly reduces the number of patterns the network needs to learn. 

Another important feature is its ability of abstracting and concentrating information. In Fig.\ref{fig:cnn}, each neuron $A$ (instance of kernel) on first layer accesses information from 4 pixels. On the second layer, each neuron $B$ connect to 4 neurons in the first layer, which means it can access information gathered from 9 pixels in the input image. As the network goes deeper, the neurons in later layers get access to larger area of the input image. At last, at the final layer, the network gets an overall abstract sense of the input image. All of these concentration and abstract procedure are learned automatically by back-propagation. It is still a mystery to researches that how those things exactly happen because the mid product of hidden layers are really difficult to understand by human beings.

\textbf{ReLU Layers} 

ReLU layers usually stand between 2 convolutional layers. ReLU stands for Rectified Linear Units. ReLU layer applies the non-saturating activation function to the outputs of convolutional layers:

\begin{equation} \label{eq:relu}
f(x) = max(0,x)
\end{equation}

ReLU layers are very simple but they efficiently add nonlinear properties to the decision
making function of the overall network as well as the sigmoid function:

\begin{equation} \label{eq:sigmoid}
f(x) = \frac{1}{1+e^{-x}}
\end{equation}

and the hyperbolic tangent function:

\begin{equation} \label{eq:hypo}
f(x) = \tanh(x)
\end{equation}

\begin{figure}[H]
  \centering
  \includegraphics[width=\linewidth]{figures/activation.png}
  \caption{Common nonlinear functions used in CNN: ReLU, Sigmoid and hyperbolic tangent.}
  \label{fig:activation}
\end{figure}

Fig.\ref{fig:activation} shows the response of 3 methods. The 3 methods share the same idea of inhibiting negative outputs and amplifying/keeping positive outputs of the early layers, which is a simulation of how human brain cells work. Hyperbolic tangent function and sigmoid function are widely used in old models but ReLU function becomes more preferable recently because it is proved to be much computational cheaper without making any significant differences in accuracy \cite{Krizhevsky2012}.

\textbf{Pooling Layers} 

\enquote{Pooling} is a nonlinear down-sampling method widely used in CNNs. Fig.\ref{fig:pool} shows a common max pooling layer with a $2 \times 2$ filter size and a stride of 2. The filter move through the entries with a certain stride, pooling layer maps each block in former layer to a single value. Pooling layer concentrates the information in former layer and provides the later layers a larger \enquote{vision} in the original image. Also, pooling helps reducing the number of parameters in the network and hence has an effect of overfit control. The most popular pooling methods are max pooling and min pooling, where the filter takes the max or min value in each block as the output. Average pooling, which uses the average of all values in the block as output is also commonly used in old days. However, it has given its place to max pooling since the later one is proved to work better in practice \cite{Scherer2010}.

\begin{figure}[H]
  \centering
  \includegraphics[width=\linewidth]{figures/pooling.png}
  \caption{A max pooling layer with a $2 \times 2$ filter size and a stride of 2}
  \label{fig:pool}
\end{figure}

\input{Chapters/Method}

\chapter{Evaluation}

\section{Evaluation Method} 

 
\textbf{Dataset:} 
This database comprises 1511 quality annotated images based on 1511 source reference image patches that are subject to different distortion levels of compression. 
Differential mean opinion score (DMOS) for this dataset were computed for each pair, which is in the range 1 to 5.

\textbf{Evaluation Metrics:}
To evaluate the performances of the IQA algorithms, we used two standard measures, i.e., Spearman's rank order correlation coefficient (SRCC) and Pearson's linear correlation coefficient (PLCC).

\textbf{Experiment Setup:}
Both the experiments in this thesis are performed on HMII database.

For the first experiment, the purpose is to evaluate how well an objective metric agrees with subjective preferences of subjects. We carefully select the Mathlab implementations of 7 algorithms to predict object scores for the entire database. 

For the second one, different models are competed to find the best \enquote*{ground truth} predictor for patch quality. 
Results reported are based on the average performance of 10 folds cross-validation. Deep learning models converge after 50 epochs.   


\section{Experiment results}

\subsection{HMII Benchmark Analysis}

First, we evaluate the pairwise preference consistency using the classic correlation coefficients SRCC and PLCC,
as shown in Table \ref{tab:algos}. 
The SRCC and PLCC are
the average values for the distorted images of the same
reference image, and the top 2 correlation coefficient values are highlighted. 
We can see that the PSNR and UQI are poorly correlated
with human perceptual quality, and even contrary to subjective
results. This defective performance of PSNR is also mentioned in the work of Zhang \emph{et al.}\cite{Zhang2019} about Fine-Grained Quality Assessment. Although VSI combine the HVS features and achieve more
consistent results than PSNR in global image assessment, it is poorly correlated with human perceptual quality in fine-grained patch quality assessment.
For the two correlation coefficients, these IQA methods
shows quite similar characteristics. As a whole, FSIM achieves top 2 performance for all the cases and the SSIM achieves better performance with PLCC while SRSIM performs better with SRCC. 

\begin{table}[ht]
  \centering
  \begin{tabular}{|l|cc|cc|}
    \hline
    \multirow{2}{*}{} & \multicolumn{2}{c|}{ HMII (64x64) }      & \multicolumn{2}{c|}{ HMII (128x128) }    \\ \cline{2-5} 
    & PLCC           & SRCC           & PLCC           & SRCC           \\ \hline
    SSIM\cite{Wang2004}             & \textbf{0.785} & 0.787          & \textbf{0.795} & 0.797          \\
    RFSIM\cite{Zhang2010}             & 0.774          & 0.757          & 0.789          & 0.759          \\
    FSIM\cite{Zhang2011}              & \textbf{0.794} & \textbf{0.799} & \textbf{0.824} & \textbf{0.815} \\
    PSNR              & 0.200          & 0.737          & 0.194          & 0.752          \\
    UQI\cite{Wang2002}               & 0.023          & 0.621          & 0.012          & 0.589          \\
    VSI\cite{Zhang2014}               & 0.765          & 0.765          & 0.768          & 0.786          \\
    SRSIM\cite{Zhang2012}             & 0.777          & \textbf{0.803} & 0.718          & \textbf{0.803} \\ \hline
  \end{tabular}
  \caption{PLCC and SRCC for different IQA algorithms}
  \label{tab:algos}
\end{table}

In addition to visualize the objective score by the top 3 IQA algorithms, we plot the distributions of subjective scores and objective scores on a 2-D graph and also plot the fitted curve on the same figure. The following figure shows the scatter distributions of subjective DMOS versus the predicted scores obtained by the SSIM, FSIM and SRSIM on the proposed database.


\begin{figure}[H]
  \centering
  \begin{subfigure}[b]{0.5\textwidth}
    \includegraphics[width=\textwidth]{figures/srsim.png}
    \caption{SRSIM}
    \label{fig:srsim}
  \end{subfigure}%
  ~ %add desired spacing between images, e. g. ~, \quad, \qquad etc.
  %(or a blank line to force the subfigure onto a new line)
  \begin{subfigure}[b]{0.5\textwidth}
    \includegraphics[width=\textwidth]{figures/fsim.png}
    \caption{FSIM}
    \label{fig:fsim}
  \end{subfigure}
  ~ %add desired spacing between images, e. g. ~, \quad, \qquad etc.
  %(or a blank line to force the subfigure onto a new line)
  \begin{subfigure}[b]{0.5\textwidth}
    \includegraphics[width=\textwidth]{figures/ssim.png}
    \caption{SSIM}
    \label{fig:ssim}
  \end{subfigure}
  \caption{Objective Score by top 3 IQA on HMII}\label{fig:animals}
\end{figure}

From the plots, we can see that these IQA algorithms tend to predict higher score for patches. 
SRSIM and FSIM frequently predict score which is higher than $0.9_{[0-1]}$ for the image with DMOS is greater than $2_{[1-5]}$.
Although FSIM achieves highest performance with the two correlation coefficients, SSIM achieves more consistent results with subjective results on the diagrams.
These
results prove that some existing IQA models perform poorly
in distinguishing the fine-grained distortion levels, which are
feasible to determine by human visual system. Therefore, these
metrics may not be suitable for perceptual-based image compression because the distortion differences between various
coding modes are usually marginal. Moreover, the fine-grained
image-patch quality assessment is demanded and should be evaluated
on the HMII databases.

\subsection{Image-Patch Models}

In this experiment, we use the following models to evaluate with our proposed DIPQA:

\begin{itemize}
  \item \textit{IPM}: Zhang \textit{et al.}\cite{Zhang2019} assume the the curve model to predict image-patch quality is a cubic polynomial function:
  $$f(\Phi(\textbf{d});\theta) = a\Phi(\textbf{d})^{3} + b\Phi(\textbf{d})^{2} + c\Phi(\textbf{d}) + d$$
  where $\theta = {a, b, c, d}$ are the parameters for the non-linear function of Image-Patch model and $\Phi(\textbf{d})$ represents the feature of patch $\textbf{d}$. MSE and SSIM are chosen for the design of features. In our work, we tried top 3 FR-IQA methods from the first experiment: SSIM, FSIM and SRSIM.  
  
  \item \textit{DIQaM}: Bosse \textit{et al.}\cite{Bosse2018} present a Deep Neural Networks for No-Reference and Full-Reference Image Quality Assessment which obtains superior performance on different IQA benchmarks. We utilize the extractor architecture from this paper to train a Deep Neural Network on our database.
\end{itemize}

First, we use the previous works to extract the SSIM, FSIM and SRSIM feature for IPM. Then, the above curve model is fitted using the least square method to obtain the parameters that best fit the learning set. DIQaM, DIPQA (VGG extractor) and DIPQA (VGG finetuning) is built with the similar architecture which share them same regression part. With the proposed DIPQA, we approach with two different tuning strategies: one is fine-tuned with the VGGNet weight and then retrained with HMII; the other one use VGGNet as a feature extractor, this part is not trained with the entire network.
  
% Second contribution
\begin{table}[ht]
  \centering
  \begin{tabular}{|l|cc|cc|}
    \hline
    \multirow{2}{*}{} & \multicolumn{2}{c|}{ HMII (64x64) } & \multicolumn{2}{c|}{ HMII (128x128) } \\ \cline{2-5} 
    & PLCC              & SRCC            & PLCC               & SRCC             \\ \hline
    IPM (SSIM)             & 0.836             & 0.784            & 0.843              & 0.794             \\
    IPM (FSIM)             & 0.848             & 0.795            & 0.871              & 0.810             \\
    IPM (SRSIM)            & 0.854             & 0.802            & 0.857              & 0.798             \\
    DIQaM                  & 0.916             & 0.824            & 0.905              & 0.819             \\
    DIPQA (VGG extractor)  & 0.802             & 0.754            & 0.830              & 0.760             \\
    DIPQA (VGG finetuning) & \textbf{0.921}    & \textbf{0.848}   & \textbf{0.955}     & \textbf{0.871}    \\ \hline
  \end{tabular}
  \caption{Comparing different Full-Reference Image-Patch approaches}
  \label{tab:approachs}
\end{table}

The Table \ref{tab:approachs} summarizes the performance of the proposed models in comparison to other methods on HMII database in terms of PLCC and SRCC. With any of the two correlation coefficients, DIPQA (VGG finetuning) achieve superior performance to the others. From the results of this project, we can also see that the larger size of the patch seem to be more accurate when assessing image-patch quality by Objective models. 

\chapter{Conclusions}

\section{Conclusions}

This project presents a new subject quality rating database considering local image quality assessment. Due to the lack of \enquote*{ground truth} quality of patches, we expect HMII to be a useful database for patch-based approaches. We also introduce a simple effective patch-based deep neural network that allows for feature learning and regression in an end-to-end framework. We believe that this proposed approach could achieve better result if we enlarge HMII database.

\section{Future work}

\textbf{HMII Database:} There are still some limitations on the proposed database to improve. 

\begin{itemize}
  \item Enlarge database to increase the number of image and them number of subject per image
  \item Generate more images to cover more type of distortions 
  \item Filter with different outlier detection methods
\end{itemize}  

\textbf{Image-Patch model:} In the future, we are planning to design more Image-Patch models and do experiments to evaluate on different databases. With DIPQA, there are some applications that we also consider: 
\begin{itemize}
  \item Applied in Image and Video Compression
  \item Associated a pooling state to compete with other Image and Video Quality Assessment (VQA) algorithms
  \item Improve current IQA/VQA algorithms
\end{itemize}  


%% ----------------------------------------------------------------
% Now begin the Appendices, including them as separate files

\addtocontents{toc}{\vspace{2em}} % Add a gap in the Contents, for aesthetics

\appendix % Cue to tell LaTeX that the following 'chapters' are Appendices

% \input{Appendices/AppendixA}	% Appendix Title

% \input{Appendices/AppendixB} % Appendix Title

% \input{Appendices/AppendixC} % Appendix Title

\addtocontents{toc}{\vspace{2em}}  % Add a gap in the Contents, for aesthetics
\backmatter

%% ----------------------------------------------------------------
\label{Bibliography}
\lhead{\emph{\mybibname}}  % Change the left side page header to "Bibliography"
\bibliographystyle{IEEEtran}  % Use the "unsrtnat" BibTeX style for formatting the Bibliography
\renewcommand{\bibname}{\mybibname}
\bibliography{References/references}


\end{document}  % The End
%% ----------------------------------------------------------------